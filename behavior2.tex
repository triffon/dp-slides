\PassOptionsToClass{aspectratio=169}{beamer}
\documentclass[alsotrans]{beamerswitch}
\usepackage{dp}

\title{Поведенчески  шаблони -- част 2}

\date{15--16 декември 2025 г.}

\hypersetup{colorlinks,urlcolor=blue,linkcolor=white}

\begin{document}

\begin{frame}
  \titlepage
\end{frame}

\begin{frame}
  \frametitle{Спомен (Memento)}

  \begin{purpose}
    Експортиране на вътрешното състояние на обект
  \end{purpose}

  \pause
  \vspace{2ex}
  \comps
  \begin{itemize}
  \item спомен (memento)
  \item създател (originator)
  \item уредник (caretaker)
  \end{itemize}
\end{frame}

\begin{frame}
  \frametitle{Команда (Command)}

  \begin{purpose}
    Капсулиране на действие като обект
  \end{purpose}

  \pause
  \vspace{2ex}
  \comps
  \begin{itemize}[<+->]
  \item абстрактна команда
  \item конкретна команда
  \item клиент
  \item изпълнител (invoker)
  \item получател (receiver)
  \end{itemize}
\end{frame}

\begin{frame}
  \frametitle{Медиатор (Mediator)}

  \begin{purpose}
    Конфигурируема връзка между обекти
  \end{purpose}

  \pause
  \vspace{2ex}
  \comps
  \begin{itemize}[<+->]
  \item абстрактен медиатор
  \item конкретен медиатор
  \item абстрактен колега
  \item конкретни колеги
  \end{itemize}
\end{frame}

\begin{frame}
  \frametitle{Итератор (Iterator)}

  \begin{purpose}
    Абстрахиране на обхождането на колекция от обекти
  \end{purpose}

  \pause
  \vspace{2ex}
  \comps
  \begin{itemize}[<+->]
  \item абстрактна колекция
  \item абстрактен итератор
  \item конкретна колекция
  \item конкретен итератор
  \end{itemize}
\end{frame}


\end{document}

\PassOptionsToClass{aspectratio=169}{beamer}
\documentclass[alsotrans]{beamerswitch}
\usepackage{dp}

\begin{filecontents*}{antipatterns.bib}
@book{brown1998antipatterns,
  title={AntiPatterns: refactoring software, architectures, and projects in crisis},
  author={Brown, William H and Malveau, Raphael C and McCormick, Hays W ``Skip'' and Mowbray, Thomas J},
  year={1998},
  publisher={John Wiley \& Sons, Inc.}
}
@misc{wikiantipatterns,
  title={Category: Anti-patterns},
  author={Wikipedia},
  note={\url{https://en.wikipedia.org/wiki/Category:Anti-patterns}}
}
@misc{sourcemakingantipatterns,
  title={AntiPatterns},
  author={Source Making},
  note={\url{https://sourcemaking.com/antipatterns}}
}
\end{filecontents*}


\title{Антишаблони за дизайн}

\date{16 януари 2024 г.}

\hypersetup{colorlinks,urlcolor=blue,linkcolor=white}

\begin{document}

\begin{frame}
  \titlepage
\end{frame}

\begin{frame}
  \frametitle{Какво са антишаблоните?}

  \begin{center}
    % https://www.pinterest.com.mx/pin/724938871241700679/
    \includegraphics[height=.7\textheight]{catvader.jpg}
    
    \LARGE \textbf{WELCOME TO THE DARK SIDE!}
  \end{center}
\end{frame}

\begin{frame}
  \frametitle{Какво са антишаблоните?}

  \begin{itemize}
  \item често срещани решения на повтарящи се проблеми
  \item интуитивни и прости за изпълнение
  \item водят до неефективност
  \item „заразни“
  \end{itemize}
\end{frame}

\begin{frame}
  \frametitle{Защо да ги познаваме?}

  \begin{itemize}
  \item да ги назоваваме
  \item да ги разпознаваме в зародиш
  \item да ги избягваме
  \item да се борим с тях
  \end{itemize}
\end{frame}

\begin{frame}
  \frametitle{Къде да четем за тях?}
  \nocite{*}
  \bibliographystyle{plain}
  \bibliography{antipatterns}
\end{frame}

\section{Антишаблони за дизайн и разработка}

\begin{frame}
  \frametitle{Антишаблони за дизайн (1)}

  \begin{itemize}[<+->]
  \item Автогенерирана печка (Autogenerated Stovepipe)
  \item Печка за предприятието (Stovepipe Enterprise)
  \item Бъркотия (Jumble)
  \item Система от печки (Stovepipe System)
  \item Вързване на гащите (Cover Your Assets)
  \item Заключване към доставчик (Vendor Lock-In)
  \item Фалшив билет (Wolf Ticket)
  \end{itemize}
\end{frame}

\begin{frame}
  \frametitle{Антишаблони за дизайн (2)}

  \begin{itemize}[<+->]
  \item Архитектура по премълчаване (Architecture By Implication)
  \item Топли тела (Warm Bodies)
  \item Дизайн от комитет (Design By Committee)
  \item Швейцарско ножче (Swiss Army Knife)
  \item Изобретяване на колелото (Reinvent The Wheel)
  \item Великият дук на Йорк (The Grand Old Duke of York)
  \end{itemize}
\end{frame}

\section{Антишаблони за разработка}

\begin{frame}
  \frametitle{Антишаблони за разработка (1)}
  
  \begin{itemize}[<+->]
  \item Петното (The Blob)
  \item Непрекъснато остаряване (Continuous Obsolence)
  \item Поток от лава (Lava Flow)
  \item Двусмислена гледна точка (Ambiguous Viewpoint)
  \item Функционална декомпозиция (Functional Decomposition)
  \item Полтъргайсти (Poltergeists)
  \item Корабна котва (Boat Anchor)
  \end{itemize}
\end{frame}

\begin{frame}
  \frametitle{Антишаблони за разработка (2)}
  
  \begin{itemize}[<+->]
  \item Златен чук (Golden Hammer)
  \item Задънена улица (Dead End)
  \item Спагети код (Spaghetti Code)
  \item Омазан вход (Input Kludge)
  \item Минно поле (Walking through a Minefield)
  \item Програмиране чрез копиране (Cut-and-Paste Programming)
  \item Отглеждане на гъби (Mushroom Management)
  \end{itemize}
\end{frame}

\end{document}
